\documentclass[../main.tex]{subfiles}
\begin{document}
\section{Введение}

\begin{Def}\label{def:fourSer}
    Пусть функция $f$ ограничена на $\Real$,  $(2\pi)$-периодична и
    интегрируема на любом конечном отрезке $[a, b] \subset \Real$.
    Тогда \mdef{рядом Фурье} для этой функции будем называть такого крокодила:
    \begin{equation}\label{eq:fourSer}
        f(x) = \frac{a_0}{2} + \sum\limits_{k=1}^{\infty}
        a_k \cos kx + b_k \sin kx
    ,\end{equation}
    где
    \begin{equation}\label{eq:coeffFourSer}
        a_k = \frac{1}{\pi}\int\limits_{-\pi}^{\pi} f(t) \cos kt \,dt \quad
        b_k = \frac{1}{\pi}\int\limits_{-\pi}^{\pi} f(t) \sin kt \,dt 
    .\end{equation} 
\end{Def} 

Сформулируем \mdef{достаточные условия Дирихле}:\\
\begin{Th}
    $f(x)$ имеет на $[-\pi, \pi]$:
\begin{enumerate}
    \item конечное число локальных экстремумов,
    \item не более чем счетное число разрывов I рода.
\end{enumerate} 
Тогда ряд~\eqref{eq:fourSer} сходится поточечно к $\frac{f(x + 0) + f(x - 0)}{2}$.
\end{Th}

Как всем известно с десткого сада, система функций
$\left\{ 1, \cos kx, \sin kx \right\}, k \in \Nat$ образует в пространстве
$L_2$ полную ортогональную систему.
Но все любят экспоненты, поэтому можно сказать, что
\begin{equation*}
    1 = e^{i0x}, \quad
    \cos kx = \frac{e^{ikx} + e^{-ikx}}{2}, \quad
    \sin kx = \frac{e^{ikx} - e^{-ikx}}{2}
.\end{equation*} 

Тогда система $\left\{ e^{ikx} \right\}, k \in \Nat$ тоже будет полной.
Пересчитаем коэффициенты:

\begin{equation*}
    \sum\limits_{k=-\infty}^{\infty} c_k e^{ikx} = \frac{a_0}{2} +
    \sum\limits_{k=1}^{\infty} a_k \cos kx + b_k \sin kx 
.\end{equation*} 

Отсюда методом пристальново взгляда получаем формулы для коэффициентов~$c_k$:
\begin{align*}\label{eq:fourSerExp}
    \begin{cases}
    c_{-k} &= (a_k + ib_k)/2, \quad k \in \Nat, \\
    c_0 &= a_0 / 2, \\
    c_k &= (a_k - ib_k)/2, \quad k \in \Nat,
    \end{cases} 
\end{align*} 

где $a_k,\ b_k$ считаются по формулам~\eqref{eq:coeffFourSer}.
Теперь немного физической интерпретации:
\begin{itemize}
    \item $\left\lvert c_k \right\rvert$~---~амплитуда комплексных гармонических колебаний,
    \item $k$~---~частота комплексных гармонических колебаний,
    \item  $\arg c_k$~---~начальная фаза.
\end{itemize} 

\end{document} 
