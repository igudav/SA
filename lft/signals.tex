\documentclass[main.tex]{subfiles}
\begin{document}
\section{Обработка сигналов}

Будем вертеть сигналами.
Сигналами удобно вертеть не как есть, а в образах Фурье.
Казалось бы, ну перегнал, ну покрутил там что-то и перегнал обратно.
Но возникают 2 проблемы:
\begin{enumerate}
    \item Интеграл в преобразовании Фурье несобственный в обе стороны.
    \item Он может не считаться аналитически, тогда придется считать его численно, а значит, мы потеряем какую-то часть информации и что-то исказим.
\end{enumerate} 
Вот мы сейчас и займемся ковырянием этих искажений и попытаемся их уменьшить.

Дисклеймер: в этой главе все суммы и интегралы понимаются формально.
Мы будем вертеть ими, как хотим, менять местами сумму и интеграл и ничего нам за это не будет.
Пусть нам поступает некоторый сигнал $f(t)$. 

Будем расматривать его дискретизацию: вектор значений $f(t)$ в некоторых точках: $f[n],\ n=\overline{1,N}$.
Точки, в которых мы знаем  значения нашей функции, будем называть \mdef{отсчетами}.
Наша цель~---~придумать такую функцию, которая бы символизировала преобразование непрерывного сигнала в дискретный и при этом хорошо преобразовывалась по Фурье.
Так с чего начать, чтобы разобрать сигнал?
Начнем с того, что введем функцию, которую будем называть \mdef{<<забором>>}
или \mdef{<<пиками>>}: 
$d_\Delta(t) \eqdef \sum\limits_{-\infty}^{\infty} \delta(t - n\Delta)$

\begin{Def}\label{def:sitOnLance}
    Функцией, \mdef{<<посаженной на пики>>} будем называть:
    \begin{equation}
        f_\Delta(t) \eqdef f(t)d_\Delta(t)
    .\end{equation} 
\end{Def}

Функция, посаженная на пики~---~это в некотором роде и есть дискретизация сигнала.
Действительно, $f_\Delta(t)$ равна нулю везде, кроме наших отсчетов,
а интеграл по сколь угодно малой окрестности отсчета будет равен значению функции в этом отсчете.

Вторая операция, которую мы рассмотрим~---~это 
\begin{Def}\label{def:convolFence}
    \mdef{<<Свертка с забором>>}
    \[
        f_0(t) \eqdef f * d_\Delta(t)
    .\] 
\end{Def} 

Давайте вычислим явно $f_0(t)$.

\[
    f_0(t) = \sum\limits_{n=-\infty}^{\infty} f * \delta(t - n\Delta) =
    \sum\limits_{n=-\infty}^{\infty} \int\limits_{-\infty}^{\infty} 
        f(\tau) \delta(t - n\Delta - \tau)\,d\tau =
    \sum\limits_{n=-\infty}^{\infty} f(t - n\Delta)
.\]

То есть <<сворачивание с забором>> равносильно <<размножению>> нашей функции
с шагом $\Delta$.

\begin{figure}[ht]
    \begin{minipage}{0.49\textwidth}
    \centering
    \incfig{sit-on-fence}
    \caption{Сигнал, <<посаженный на пики>>}
    \label{fig:sit-on-fence}
    \end{minipage}
    \begin{minipage}{0.49\textwidth}
    \centering
    \incfig{convol-with-fence}
    \caption{Сигнал, свернутый с забором}
    \label{fig:convol-with-fence}
    \end{minipage} 
\end{figure}

\end{document}
