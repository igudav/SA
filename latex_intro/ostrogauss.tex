\documentclass[12pt, a4paper]{article} % свойства докуменат
\usepackage[utf8]{inputenc} % хотим нормальную кодировку
\usepackage[T2A]{fontenc} % тип шрифта, по-моему
\usepackage[russian]{babel} % русские буквы и обозначения
\usepackage{graphicx, xcolor} % графика
\usepackage{subfiles} % царская разбивка на много файлов
\usepackage{amsmath} % различные нужные символы, типа \geqslant
\usepackage{amsthm} % для окружений типа theorem
\usepackage{amssymb} % еще немного символов
\usepackage{esint} % стили для интегралов, позволяет писать \oiint
\usepackage{import} % для включения рисунков

% комманда для царского добавления в документ векторной графики
\newcommand{\incfig}[1]{%
    \def\svgwidth{\columnwidth}
    \import{figures/}{#1.pdf_tex}
}
\pdfsuppresswarningpagegroup=1

% перенос во включенных формулах
\newcommand*{\hm}[1]{#1 \nobreak\discretionary{}%
    {\hbox{$\mathsurround=0pt #1$}}{}}

% комманда для частной производной
\newcommand{\dd}[2]{\frac{\partial#1}{\partial#2}}

% русские знаки нестрогих неравенств
\renewcommand{\le}{\leqslant}
\renewcommand{\ge}{\geqslant}

% настраиваем окружения
\newtheorem{Th}{Теорема}
\newenvironment{Proof}{\par\textbf{Доказательство. }}
	{\hfill$\blacksquare$\vspace{0.1cm}} 

\begin{document}

\subfile{titul.tex}

\section*{Формула Остроградского~-- Гаусса}

Пусть $D$~--- односвязная область в $E^3$ (т.\,е. для любой кусочно-гладкой
замкнутой кривой $C$ можно указать ориентированную кусочно-гладкую поверхность
$G$, расположенную в $D$, для которой $C$ является границей).
Пусть $S = \partial D$~--- граница этой области, удовлетворяющая условиям:
\begin{enumerate}
    \label{cond1}
    \item Поверхность $S$~--- кусочно-гладкая двусторонняя полная ограниченная
        замкнутая и без особых точек.
    \label{cond2}
    \item Прямоугольную декартову систему координат (далее ПДСК) можно выбрать
        так, что для каждой из осей координат любая прямая, параллельная этой
        оси, будет пересекать поверхность $S$ не более чем в двух точках.
\end{enumerate} 

Пусть $\vec{n}$~--- единичный вектор внешней нормали к $S$.
Справедлива следующая теорема.

\begin{Th}[формула Остроградского~--Гаусса]
    Пусть $\vec{a}$~--- векторное поле, дифференцируемое в области $D$,
    удовлетворяющей условиям 1 и 2, и такое что производная по любому 
    направлению непрерывна в $D \cup \partial D \hm= \overline{D}$.
    Тогда справедлива формула
    \begin{equation}\label{eq:ostrogauss}
        \underbrace{
            \iiint\limits_{\overline{D}} \vec{a} \, dv 
        }_{\text{\parbox{2cm}{\scriptsize \centering интеграл от дивергенции}}} = 
        \underbrace{
            \oiint\limits_{\partial D} \left<\vec{a}, \vec{n} \right>\, ds
        }_{\text{\parbox{2cm}{\scriptsize \centering поток поля через поверхность}}}
    .\end{equation} 
    Коротко: интеграл от дивергенции равен потоку.
\end{Th} 

\begin{Proof}
    Все входящие в формулу~\eqref{eq:ostrogauss} функции непрерывны,
    поэтому интегралы в обеих частях равенства существуют.

    Заметим, что формула~\eqref{eq:ostrogauss} инвариантна относительно
    выбора ПДСК, так как инвариантно все, что входит в эту формулу.
    Поэтому достаточно доказать теорему для какой-то одной ПДСК.
    Выберем ПДСК так, чтобы выполнялось условие 2.

    Пусть $\vec{a} = \left( P, Q, R \right)^T$, 
    $\vec{n} = (\cos \alpha, \cos \beta, \cos \gamma)^T$.
    Тогда, учитывая, что
    \begin{equation*}
        \cos \alpha\, ds = dydz, \quad
        \cos \beta\, ds = dzdx, \quad
        \cos \gamma\, ds = dxdy, \quad
    \end{equation*} 
    
    \noindent
    получим

    \begin{multline}\label{eq:ostrogauss1}\tag{1'}
        \iiint\limits_{\overline{D}} \left( 
            \dd Px + \dd Qy + \dd Rz
        \right)\, dxdydz = \\
        = \oiint\limits_{\partial D} \bigl( 
            P\cos\alpha + Q\cos\beta + R\cos\gamma
        \bigr)\, ds = \\
        = \oiint\limits_{\partial D} \bigl( 
            P\,dydz + Q\,dzdx + R\,dxdy
        \bigr)
    .\end{multline} 

    \noindent
    Теперь покажем, что

     \begin{equation*}
         I = \iiint\limits_{\overline{D}} \dd Rx \, dxdydz =
         \oiint\limits_{\partial D} R\,dxdy,
    \end{equation*} 

    \begin{figure}
        \centering
        \incfig{ff}
        \caption{К доказательству теоремы.}
        \label{fig:ff} % label строго после caption!
    \end{figure} 

    \noindent
    для остальных двух пар интегралов показывается аналогично.
    Обозначим через $D'$ проекцию области  $D$ на плоскость  $Oxy$.
    Через граничные точки  $D'$ проведем прямые, параллельные оси  $Oz$.
    Каждая из этих прямых пересекается с  $\partial D$ ровно в одной точке.
    Множество всех таких точек разделяет $\partial D$ на 2 части,
    которые мы обозначим через  $S_1$ и  $S_2$ (см. рис.~\ref{fig:ff}).
    Если мы проведем прямую из внутренности $D'$, параллельную  $Oz$, 
    то она пересечет поверхность в 2 точках:  $\bigl( x, y, z_1(x, y) \bigr) \in S_1$ и $\bigl( x, y, z_2(x, y) \bigr) \in S_2$,
    причем $z_1(x, y) \ge z_2(x, y)$.
    Так как граница области кусочно-гладкая, то $z_1(x, y)$ и $z_2(x, y)$~---
    кусочно-гладкие функции в $D'$.
    По формуле сведения тройного интеграла к повторному получаем:

    \begin{multline*}
        I = \iint\limits_{D'} \left[ 
            \int\limits_{z_2(x, y)}^{z_1(x, y)} \dd{R(x, y, z)}{z}\,dz
        \right]\,dxdy =
        \iint\limits_{D'} R\bigl(x, y, z_1(x, y)\bigr)\,dxdy - \\
        - \iint\limits_{D'} R\bigl(x, y, z_2(x, y)\bigr) =
        \iint\limits_{S_1} R(x, y, z)\,dxdy + \\
        + \iint\limits_{S_2} R(x, y, z)\,dxdy =
        \oiint\limits_S R(x, y, z)\,ds.
    \end{multline*} 
    
    \noindent
    Здесь мы воспользовались тем, что $S = S_1 \cup S_2$ и соотношением

    \begin{equation*}
        - \iint\limits_{D'} R\bigl(x, y, z_2(x, y)\bigr) =
        \iint\limits_{S_2} R(x, y, z)\,dxdy =
        \iint\limits_{S_2} R\cos\gamma \, ds,
    \end{equation*} 

    \noindent
    справедливым в силу того, что внешняя нормаль $\vec{n}$ к поверхности $S_2$
    образует тупой угол с осью $Oz$, поэтому  $\cos\gamma < 0$.

\end{Proof} 
    
\end{document} 
