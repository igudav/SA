\documentclass[12pt, a4paper]{article} % свойства докуменат
\usepackage[utf8]{inputenc} % хотим нормальную кодировку
\usepackage[T2A]{fontenc} % тип шрифта, по-моему
\usepackage[russian]{babel} % русские буквы и обозначения
\usepackage{graphicx, xcolor} % графика
\usepackage{subfiles} % царская разбивка на много файлов
\usepackage{amsmath} % различные нужные символы, типа \geqslant
\usepackage{amssymb} % еще немного символов
\usepackage{import} % для включения рисунков
\usepackage{xifthen}
\usepackage{pdfpages}
\usepackage{transparent}
\usepackage{titlesec} % для настройки заголовков и секций вообще
\usepackage{caption} % для подписей к рисункам на 2 строки
\usepackage[outdir=./figures/]{epstopdf}
\usepackage{multicol}
\usepackage{float}
\usepackage{mathrsfs}

% комманда для царского добавления в документ векторной графики
\newcommand{\incfig}[1]{%
    \def\svgwidth{\columnwidth}
    \import{figures/}{#1.pdf_tex}
}
\pdfsuppresswarningpagegroup=1

\newcommand\eqdef{\stackrel{\text{\tiny def}}{=}}

\newtheorem{Th}{Теорема}
\newtheorem{Lem}{Лемма}
\newenvironment{Proof}{\par\noindent{\bf Доказательство.}}{\hfill$\scriptstyle\blacksquare$}

% русские знаки нестрогих неравенств
\renewcommand{\le}{\leqslant}
\renewcommand{\ge}{\geqslant}
\renewcommand{\emptyset}{\varnothing}
\renewcommand{\phi}{\varphi}
\renewcommand{\epsilon}{\varepsilon}

\newcommand{\Real}{\mathbb{R}}
\newcommand{\inner}[2]{\bigl< #1, #2 \bigr>}

\newcommand*{\hm}[1]{#1\nobreak\discretionary{}%
            {\hbox{\mathsurround=0pt #1}}{}}

\titleformat{\section}{\normalfont\Large\bfseries}{\thesection.}{1em}{}
\titleformat{\subsection}{\normalfont\Large\bfseries}{\thesubsection.}{1em}{}

\DeclareMathOperator{\conv}{conv}
\DeclareMathOperator*{\argmax}{argmax}
\DeclareMathOperator*{\Argmax}{Argmax}
\DeclareMathOperator{\sgn}{sgn}

% \counterwithin{section}{part}

\graphicspath{~/sa/oc_prac/lab_1/report/figures}

\begin{document}

\subfile{titul.tex}

\tableofcontents

\newpage

\part{Теоретическая часть}

\section{Постановка задач}

Задано обыкновенное дифференциальное уравнение:
\begin{equation}\label{eq:init_prob}
    \ddot x - x^2 \sin x + x^3 + x^{4} + 2 \dot x = u,
\end{equation} 
где $x \in \Real$,  $u \in [-\alpha, \alpha]$.
В начальный момент времени $x(0) = \dot x(0) \hm= 0$. 
Необходимо построить множество достижимости $X(t, t_0, x(t_0), \dot x(t_0))$
в классе программынх управлений в заданный момент времени $t \ge t_0$.

Сведем данное дифференциальное уравнение к системе уравнений 1-го порядка:
\begin{equation}\label{eq:init_sys}
    \begin{cases}
        \dot x_1 = x_2, \\
        \dot x_2 = u - 2x_2 -x_1^3 - x_1^{4} + x_1^2 \sin x_1.
    \end{cases}
\end{equation} 
Начальные условия: $x_1(0) = 0,\ x_2(0) = 0$.

\section{Вспомогательные утверждения}

Сформулируем принцип максимума Понтрягина для задач достижимости.
Он доказан в~\cite{Komar}

\begin{Th}[ПМП для задачи достижимости]\label{th:pmp}
    Рассматривается следующая автономная система:
    \begin{equation}\label{eq:pmp_sys}
        \dot x = f(x(t), u(t)).
    \end{equation} 
    Дополнительно предполагаем, что функции $f$ и $f'_x$ непрервыны,
    и ограничения на управление и не зависят от времени:
    $u(t) \in \mathcal{P} \subset \Real^{m}\ \dot \forall t$.
    Введем функцию Гамильтона-Понтрягина:
    \begin{equation}\label{eq:pmp_ham}
        \mathcal{H} \bigl( \psi(t), x(t), u(t) \bigr) = 
        \inner{\psi(t)}{f( x(t), u(t) )}.
    \end{equation} 
    Тогда если $(u^*, x^*)$"---~оптимальная пара,
    такая что  $x^*(t) \in \partial X(t)$, то существует функция $\psi^*(t)$,
    удовлетворяющая сопряженной системе:
    \begin{equation}\label{pmp_conj}
        \dot \psi = -\frac{\partial \mathcal{H}}{\partial x},
    \end{equation} 
    для которой выполнено условия максимума:
    \begin{equation}\label{eq:pmp_pmp}
         \mathcal{H} \bigl(\psi^*(t), x^*(t), u^*(t)\bigr) = 
         \sup\limits_{v \in \mathcal{P}} \mathcal{H} \bigl(\psi^*(t), x^*(t), v\bigr).
    \end{equation} 
\end{Th} 
Эта теорема позволяет использовать для построения границы множества
достижимости только те траектории, которые удовлетворяют условию максимума~\eqref{eq:pmp_pmp}.

Сформулируем также теорему, которая позволяет указать структуру переключений управления в нашей системе.
Доказательство приведено в~\cite{Komar}.
\begin{Th}[о нулях $x_2$ и $\psi_2$]
    Пусть $t_1 < t_2$, $x_2(t_1) = x_2(t_2) = 0$, $x_2$ не обращается в нуль 
    на $(t_1, t_2)$ и $\psi_2(t_1) \neq 0$.
    Тогда $\psi_2(t_2) \neq 0$ и существует $\tau \in (t_1, t_2)$, 
    такое что $\psi_2(\tau) = 0$.
\end{Th} 
Смысл теоремы в том, что между двумя моментами времени, в который $x_2$ обращается в $0$,
существует момент времени, в который  $\psi_2$ обращается в $0$.

\section{Исследование задачи}

Запишем для заданной задачи функционал Гамильтона-Понтрягина и сопряженную систему:
\begin{equation}\label{eq:prb_ham}
    \mathcal{H} = \psi_1 x_2 + \psi_2 \left( u - 2x_2 -x_1^3 - x_1^{4} + x_1^2 \sin x_1 \right)
\end{equation} 
\begin{equation}\label{eq:prb_conj}
    \begin{cases}
        \dot \psi_1 = \psi_2 \left( 3x_1^2 + 4x_1^3 - 2x_1\sin x_1 - 
            x_1^2\cos x_1\right), \\
        \dot \psi_2 = -\psi_1 + 2\psi_2. 
    \end{cases} 
\end{equation} 

Из функционала Гамильтона-Понтрягина выпишем вид экстремальных управлений:
\begin{equation}\label{prb_controls}
    u^* = 
    \begin{cases}
        \alpha, & \psi_2 > 0, \\
        [-\alpha, \alpha], & \psi_2 = 0, \\
        -\alpha, & \psi_2 < 0.
    \end{cases} 
\end{equation} 

\begin{Lem}
    Особый режим в данной задаче невозможен.
\end{Lem} 
\begin{Proof}
    Будем доказывать от противного: предположим, что $\psi_2 = 0$ на некотором интервале.
    Тогда и $\dot \psi_2 = 0$ на этом интервале.
    Тогда в силу 2-го уравнения системы~\eqref{eq:prb_conj} $\psi_1 = 0$ на этом интервале. 
    Получаем противоречие с условием нетривиальности сопряженных переменных.
\end{Proof} 

Таким образом, система движется в одном из двух режимов:
\begin{equation}\label{prb_pos}\tag{$+$}
    \begin{cases}
        \dot x_1 = x_2, \\
        \dot x_2 = \alpha - 2x_2 -x_1^3 - x_1^{4} + x_1^2 \sin x_1.
    \end{cases}
\end{equation} 

\begin{equation}\label{prb_neg}\tag{$-$}
    \begin{cases}
        \dot x_1 = x_2, \\
        \dot x_2 = -\alpha - 2x_2 -x_1^3 - x_1^{4} + x_1^2 \sin x_1.
    \end{cases}
\end{equation} 

\bibliographystyle{utf8gost705u}
\bibliography{biblio}

 \end{document} 
